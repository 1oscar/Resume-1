\documentclass[11pt,a4paper,sans]{moderncv}   % possible options include font size ('10pt', '11pt' and '12pt'), paper size ('a4paper', 'letterpaper', 'a5paper', 'legalpaper', 'executivepaper' and 'landscape') and font family ('sans' and 'roman')

% moderncv 主题
\moderncvstyle{classic}                        % 选项参数是 ‘casual’, ‘classic’, ‘oldstyle’ 和 ’banking’
\moderncvcolor{blue}                          % 选项参数是 ‘blue’ (默认)、‘orange’、‘green’、‘red’、‘purple’ 和 ‘grey’
%\nopagenumbers{}                             % 消除注释以取消自动页码生成功能

% 字符编码
\usepackage[utf8]{inputenc}                   % 替换你正在使用的编码
\usepackage{CJKutf8}

% 调整页面
\usepackage[scale=0.88]{geometry}
%\setlength{\hintscolumnwidth}{3cm}           % 如果你希望改变日期栏的宽度

% 个人信息
\firstname{\qquad廖祥俐}
\familyname{}
\title{}                     % 可选项、如不需要可删除本行
\address{湖北武汉--华中科技大学--南二舍}{邮编:430074}            % 可选项、如不需要可删除本行
\phone[mobile]{+86 13554469447}              % 可选项、如不需要可删除本行
\email{liaoxl2012@gmail.com}                    % 可选项、如不需要可删除本行
\homepage{liaoxl.github.io}                  % 可选项、如不需要可删除本行

%----------------------------------------------------------------------------------
%            内容
%----------------------------------------------------------------------------------

\begin{document}
\begin{CJK}{UTF8}{gbsn}                       % 详情参阅CJK文件包
\maketitle

\section{教育背景}
\cventry{2012--至今}{硕士(保研)}{华中科技大学自动化学院}{图像所}{将于 2015 年毕业}{}{}
\cventry{2008--2012}{本科}{华中科级大学电信系}{提高班}{}{}

\section{奖项}
\cventry{2012.6}{华中科技大学2012届 优秀毕业生}{}{}{}{}
\cventry{2011.7}{华科电信系"TI杯"电子设计大赛二等奖}{}{}{}{}
\cventry{2011.5}{第四届文鼎创杯华中地区大学生数学建模邀请赛三等奖}{}{}{}{}
\cventry{2011.3}{华科点团队首届“OPhone杯"移动应用大赛第一名}{}{}{}{}

\section{技能}
\cventry{英语}{CET-4 563 分 CET-6 529 分}{}{}{}{}
\cventry{计算机}{全国计算机技术与软件专业技术资格软件设计师中级}{}{}{}{}

\section{项目经历}
\renewcommand{\baselinestretch}{1.2}

\cventry{2013--2014}
{星形形状约束的自动目标分割}
{Matlab,C++}
{研究项目}{}
{本研究项目旨在完成对自然图像中目标的自动分割。通过定义自然场景图像中目标的通用形状“星形”,构建目标能量函数,并通过Graphcut优化,为了快速准确的确定目标的形状中心,我们通过将其与当前热门的图像显著性相结合,将显著性区域的中心点作为目标的星形中心,从而实现了具有星形信息的自动目标分割。作为主要贡献者,该\textbf{\href{http://files.cnblogs.com/moondark/XiangliLiao_SRAPC.pdf}{研究成果}}已投递Signal Processing杂志。}
\vspace*{0.2\baselineskip}

\cventry{2013}
{协同扩散分割}
{Matlab}
{研究项目}{}
{协同扩散分割旨在对图像集合中的多类目标进行统一分割。该算法基于物理学的热扩散模型,将图像看成一个传导网络,用K-近邻构建传导权值图,为了减少计算量,采用超像素+聚类进行算法优化,作为主要贡献者,该\textbf{\href{http://files.cnblogs.com/moondark/XiangliLiao_CoDiffusion.pdf}{研究成果}}已投递IEEE Trans. Multimedia杂志。}
\vspace*{0.2\baselineskip}

\cventry{2012--2013}
{手绘识别}
{C++,MFC,Linux}
{合作项目}{}
{手绘识别旨在识别由用户一笔绘制的基元图形,并在此基础上对由基元图形组成的复杂图形进行判断识别。我在该项目中负责设计复杂图形的识别规则,并应项目要求,完成对Windows/Linux平台下共享库的开发,同时负责项目的维护工作。基元图形主要通过点集的几何特征(最小凸包/最小外接矩形/最大内接三角形/最大内接四边形)之间的比例,通过支撑向量机(SVM)训练分类器进行识别,复杂图形主要通过编码及空间关系约束进行判断。}
\vspace*{0.2\baselineskip}

\cventry{2012--2014}
{四色标记分割}
{C++,MFC}
{研究项目}{}
{四色标记分割旨在自动完成对图像的多类分割。我主要负责对导师\textbf{\href{http://www.sciencedirect.com/science/article/pii/S0031320311001907}{已有成果}}(已发表在Pattern Recognition杂志)的算法实现及算法的系统级演示。多层图优化算法得到的误差与多层图自身的层数成正比,为了减小误差,根据四色定理,四种颜色可以完成对平面图所有区域的着色,从而可以将多层图压缩在四层,约束了图像的自动分割误差的上界,目前仍在探索并优化该算法。}
\vspace*{0.2\baselineskip}

\cventry{2012}
{多层图优化算法}
{C++}
{独立项目}{}
{多层图优化算法旨在优化图像分割中的多类标记问题,快速实现能量函数的优化。该算法由我导师提出,并发表了多篇研究成果,我的主要工作在于对多层图算法代码的重构,并给出了算法实现的\textbf{\href{https://github.com/liaoxl/MultiLayerGraph}{开源版本}}。算法基于开源库\textbf{\href{http://vision.csd.uwo.ca/code/}{Max-flow/min-cut}},为双向广搜的最大流算法实现。}
\vspace*{0.2\baselineskip}

\cventry{GitHub}
{\href{https://github.com/liaoxl}{https://github.com/liaoxl}}
{}{}{}{}


\renewcommand{\baselinestretch}{1.0}




\closesection{}                   % needed to renewcommands
\renewcommand{\listitemsymbol}{-} % change the symbol for lists
\clearpage\end{CJK}
\end{document}
